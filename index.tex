\input texrc
\title CLiiScm

CLiiScm ports select Common Lisp files in the working
directory to Racket.

Currently, it is used to create the Racket version of the Common
Lisp source for
\urlh{https://github.com/ds26gte/tex2page}{TeX2page}, to avoid the
tedium of maintaining two (each about 10 kLoc) sources. The Racket
version can be further converted to other Scheme dialects using
\urlh{https://github.com/ds26gte/scmxlate}{Scmxlate}.

CLiiScm isn’t a completely general-purpose translator.
Specifically, it requires uses of the Common Lisp `loop` macro to
be such that any `return` expressions occur only in a tail call
position of an immediate subform of the `loop`. A more general
translation of `loop` would have made the more common uses
unnecessarily inefficient.

Specify the files to be ported (without quotes) in the file
`cliiscm-files-to-be-ported.lisp` in the working directory.  In
the following, assume `jobname.lisp` to be a file to be ported.

Start Common Lisp and the load the file `clliscm.lisp` (using the
appropriate relative or full pathname). When clliscm is done, it
will  generally leave for each `jobname.lisp` the ported
equivalent `my-jobmame.lisp`. (You can use clliscm directives to
change the name of the output file(s).)

For each file `jobname.lisp` to be ported, a user override file
`cliiscm-jobname.lisp` can be provided in the working directory.

In the user override file, the following directives can be used
to guide the porting process:

(i) `(cliiscm-rename (old1 new1) ...)` will change all
occurrences of `old1 ...` to `new1 ...`.

(ii) `(cliiscm-rename-def (old1 new1) ...)` will ignore the
definitions for `old1 ...` in `jobname.lisp` and will change all
other occurrens of them to `new1 ...`.

(iii) `(clisscm-ignore-def name1 ...)` will ignore the
definitions for `name1 ...`.

(iv) `(cliiscm-uncall name1 ...)` will ignore all top-level calls
to `name1 ...`.

(v) `(cliiscm-defsetf (getter1 setter1) ...)` will associate
`setter1 ...` as the setter procedures for `getter1 ...`.

(vi) `(cliiscm-insert x ...)` will insert `x ...` at the head of
the output file.

(vii) `(cliiscm-postamble)` will position the contents of the
source file at that point, so any subsequent directives will go
after that.

(viii) `(cliiscm-postprocess x ...)` will perform the actions `x
...` after the output file has been created. (This can be used
to rename or change the permissions of the output file.)

(ix) Any definitions in the user definition file will
automatically cause definitions of the same names in the source
file to be ignored.

\bye
